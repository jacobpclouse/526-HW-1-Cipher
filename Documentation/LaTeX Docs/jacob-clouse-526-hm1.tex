\documentclass{article}
\author{Jacob Clouse}
\title{ICSI 526 - Spring 2023 - Homework 1}


\usepackage[margin=0.5in]{geometry}
%\usepackage{amsmath}
\usepackage{mathtools}


\begin{document}
	
\maketitle	
	
\section{\underline{ - Question 1a Answer:}}
First of all, this IS breakable. Here is my explanation of why: \\

For my First name: J A C O B, I found that the combine total is 9 + 0 + 2 + 14 + 1. (9 + 0 + 2 + 14 + 1) is equal to 26, so we use (26) mod 26 which is equal to 0. So in C1 = (a * P1 + b) mod 26, C1 is equal to 0 (for the most common letter E). E is normally valued at 4 in plaintext. \newline


For my Last name: C L O U S E, I found that the combine total is 2 + 11 + 14 + 20 + 18 + 4. (2 + 11 + 14 + 20 + 18 + 4) is equal to 69, so we use (69) mod 26 which is equal to 17. So in C2 = (a * P2 + b) mod 26, C2 is equal to 17 (for the second most common letter T). T is normally valued at 19 in plaintext. \newline

Here are the equations: \newline
For E / First name: \textbf{C1 = (a * P1 + b) mod 26} \textit{OR} \textbf{0 = (a * 4 + b) mod 26} \newline
For T / Last name:  \textbf{C2 = (a * P2 + b) mod 26} \textit{OR} \textbf{17 = (a * 19 + b) mod 26} \newline

To find the difference between the two we can subtract the first from the second: 
\begin{align}
    17&=(a * 19 + b) mod 26 \label{eqn:Last-Name}\\  
	0&=(a *  4 + b)mod26 \label{eqn:First-Name} 
\shortintertext{Subtracting \eqref{eqn:First-Name} from \eqref{eqn:Last-Name} yields:} 
	\textbf{17 = 15a  mod  26}  
\end{align}


\textit{We now need to take this function and solve for a}. To do this, we need to move the mod operator over in (3) to the left hand side. We now have: 
\begin{align}
17 mod 26 = 15a
\end{align}
% BELOW: ASK HIM IF YOUR EXPLAINATION IS CORRECT!!!!
We use the Euclidean Algorithm to find the Greatest Common Divisor (or GCD) of 15 and 26 and check to see if its equal to 1. It turns out that the GCD between 15 and 26 is 1. 
\begin{align}
\shortintertext{So this becomes:}
17*15^{-1} mod 26 = 1
\shortintertext{Then:}
17*7 mod 26 = a \\
119 mod 26 = a
\shortintertext{Finally, we find that:}
\textbf{a = 15}\end{align}
\\
\\
\textit{Now we need to solve for b.} We do this by substituting in our a value for one of our equations:\newline
\begin{align}
17 = (\textbf{15} * 19 + b) mod 26
\shortintertext{Then:}
17 = 285 + b mod 26\\
(17 - 285 ) mod 26 = b\\
-268 mod 26\\
\shortintertext{Finally, we find that:}
\textbf{b = 18}
\end{align}

\textit{To check our work} we need to substitute b into the equation and solve it:
\begin{align}
	(15(19) + 18) mod 26
\end{align}
This is equal to 17, which is the value we calculated previously. So it works!
\\
\\
\\
\section{\underline{ - Question 1b Answer:}}
\iffalse
COMMENT:
1b) if it it is part of the key, than it is not solvable
if it is a constant, then it could be solvable
even if is a known value, there could be 3 equations
THIS ONE IS SUBJECTIVE!!!!

EMAIL HIM AND ASK HIM IF THIS EXPLANATION IS GOOD!!!!!
\fi
The answer to 1b depends on if the \textbf{d} in \textbf{C = [a × (P-d) + b] mod 26} a constant or part of the key.
\begin{itemize}
	\item i) If this just a constant being added in, we \textbf{CAN} crack this! It basically, it would be similar to the offset that is already being conducted on the on the plaintext. We could use the two equations to mathematically solve for it like we did with a and b.

	\item ii) If this is part of the key, we \textbf{CAN NOT} crack this. If it was something like a One Time pad, the encryption key is a random number and, the key is used only once. That would mean we couldn't solve for it like we did for a and b, it would have no correlation between the two equations. 
\end{itemize}

\section{\underline{ - Question 2 Answer (DESIGN):}}
The design of my algorithm can be broken down into three parts: 

\begin{itemize}
	\item Substitution Section
	\item Transposition Section
	\item One Time Pad Section
\end{itemize}
I originally had coded each piece separately, tested them individually and then combined them into one overarching program (the final Product Cipher).
The first thing I do inside of my Product Cipher is to get the plaintext that the user wants to encrypt. Once they enter it in, we move onto the \textbf{substitution section}. 
The substitution section is a spin on the classic Caesar/Affine cipher we have gone over in class. We map values for all the letters of the alphabet to number values so we can then do a displacement. We translate our plaintext into number form and ask the user to give us the displacement amount. We then combine the plaintext value and the offset together, then modulus it by 26. We return this value out of the function and feed it into the \textbf{transposition section}). \newline
The transposition section is loosely based off of the row transposition cipher. The user gives us input, that input is then stored into a 2D matrix and then they choose a key for this matrix. For simplicity's sake, I limited my cipher to have 4 columns in total. This would potentially cut down on the security of the algorithm, but combining this section with the other two more than makes up for it. We receive the substitution ciphertext directly from the code and then move it into our 2D matrix. The length of the plaintext has to be a multiple of 4, so we have to add padding if it is not. We then query the user to give us the column order key by providing us with a combination of 1,2,3 and 4. This is then used to reorganize the matrix and transpose it back into a string, giving us our transposition ciphertext. The only thing left to do is the \textbf{one time pad section}.\newline
What makes this cipher truly secure is the one time pad section. The one time pad as a concept was created over 100 years ago and it is still being used in cryptography to this very day because it is so secure. We use our transposition ciphertext as the input to our one time pad function and it determines the number its length. It then goes and generates a random value between 0 and 9 for each character. This gets us a string of numbers the same length of our key. We then combine this key with our plaintext using ord() operations. This should only be breakable if you have the specific key that goes along with your ciphertext. Otherwise, this ensures the security of the full cipher. Our cipher and key are stored as pickle objects and printed out in our terminal. Separate the key from the cipher, and it should be pretty dang secure. \newline
\newline
To decrypt our cipher, all we need to do is reverse our process. First, we start off with the one time pad section, you just need to provide the pickle files for the key and the ciphertext. It will open up the pickle files and reverse the operation (done the same way you encrypted them). This is followed by the transposition section. The output from the one time pad decrypt is fed into the transposition decrypt function, all it needs from you is the column combination key that you set when you encrypted it. Once you provide that, it will rearrange the columns back to their decrypted state. But we are not done, we just have to do the substitution cipher decrypt. The output from the transposition decrypt is fed once again into the substitution cipher decrypt function, all we need to do is provide it the offset that you originally set during the encryption. And, just like that, we are done with the decryption!

%When you combine all of these into one, you get your product cipher!



\section{\underline{ - Question 2a Answer:}}
(a) breaking your algorithm is going to require coding effort (i.e., your algorithm cannot be broken by
using pen and paper).

\section{\underline{ - Question 2b Answer:}}
(b) your algorithm is secure against any two cryptanalytic attacks.


\end{document}