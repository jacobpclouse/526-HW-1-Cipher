\documentclass{article}
\author{Jacob Clouse}
\title{ICSI 526 - Spring 2023 - Homework 1}


\usepackage[margin=0.5in]{geometry}
%\usepackage{amsmath}
\usepackage{mathtools}


\begin{document}
	
\maketitle	
	
\section{\underline{ - Question 1a Answer:}}
First of all, this IS breakable. Here is my explanation of why: \\

For my First name: J A C O B, I found that the combine total is 9 + 0 + 2 + 14 + 1. (9 + 0 + 2 + 14 + 1) is equal to 26, so we use (26) mod 26 which is equal to 0. So in C1 = (a * P1 + b) mod 26, C1 is equal to 0 (for the most common letter E). E is normally valued at 4 in plaintext. \newline


For my Last name: C L O U S E, I found that the combine total is 2 + 11 + 14 + 20 + 18 + 4. (2 + 11 + 14 + 20 + 18 + 4) is equal to 69, so we use (69) mod 26 which is equal to 17. So in C2 = (a * P2 + b) mod 26, C2 is equal to 17 (for the second most common letter T). T is normally valued at 19 in plaintext. \newline

Here are the equations: \newline
For E / First name: \textbf{C1 = (a * P1 + b) mod 26} \textit{OR} \textbf{0 = (a * 4 + b) mod 26} \newline
For T / Last name:  \textbf{C2 = (a * P2 + b) mod 26} \textit{OR} \textbf{17 = (a * 19 + b) mod 26} \newline

To find the difference between the two we can subtract the first from the second: 
\begin{align}
    17&=(a * 19 + b) mod 26 \label{eqn:Last-Name}\\  
	0&=(a *  4 + b)mod26 \label{eqn:First-Name} 
\shortintertext{Subtracting \eqref{eqn:First-Name} from \eqref{eqn:Last-Name} yields:} 
	\textbf{17 = 15a  mod  26}  
\end{align}


\textit{We now need to take this function and solve for a}. To do this, we need to move the mod operator over in (3) to the left hand side. We now have: 
\begin{align}
17 mod 26 = 15a
\end{align}
% BELOW: ASK HIM IF YOUR EXPLAINATION IS CORRECT!!!!
We use the Euclidean Algorithm to find the Greatest Common Divisor (or GCD) of 15 and 26 and check to see if its equal to 1. It turns out that the GCD between 15 and 26 is 1. 
\begin{align}
\shortintertext{So this becomes:}
17*15^{-1} mod 26 = 1
\shortintertext{Then:}
17*7 mod 26 = a \\
119 mod 26 = a
\shortintertext{Finally, we find that:}
\textbf{a = 15}\end{align}
\\
\\
\textit{Now we need to solve for b.} We do this by substituting in our a value for one of our equations:\newline
\begin{align}
17 = (\textbf{15} * 19 + b) mod 26
\shortintertext{Then:}
17 = 285 + b mod 26\\
(17 - 285 ) mod 26 = b\\
-268 mod 26\\
\shortintertext{Finally, we find that:}
\textbf{b = 18}
\end{align}

\textit{To check our work} we need to substitute b into the equation and solve it:
\begin{align}
	(15(19) + 18) mod 26
\end{align}
This is equal to 17, which is the value we calculated previously. So it works!



\section{\underline{ - Question 1b Answer:}}
For the 
\newline
\newline
For the 
\newline

\section{\underline{ - Question 2a Answer:}}
For the 
\newline
\newline
For the 
\newline

\section{\underline{ - Question 2b Answer:}}
For the 
\newline
\newline
For the 
\newline


\end{document}